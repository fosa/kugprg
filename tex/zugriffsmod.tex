% coding:utf-8

%kugprg, a LaTeX-Code for a summary of programming modules
%Copyright (C) 2013, Daniel Winz, Ervin Mazlagic

%This program is free software; you can redistribute it and/or
%modify it under the terms of the GNU General Public License
%as published by the Free Software Foundation; either version 2
%of the License, or (at your option) any later version.

%This program is distributed in the hope that it will be useful,
%but WITHOUT ANY WARRANTY; without even the implied warranty of
%MERCHANTABILITY or FITNESS FOR A PARTICULAR PURPOSE.  See the
%GNU General Public License for more details.
%----------------------------------------


\newpage
\section{Zugriffsmodifizierer}

% http://stackoverflow.com/questions/215497/in-java-whats-the-difference-between-public-default-protected-and-private
\begin{table}[h!]
	\centering
	\begin{tabular}{l c c c c}
			& Class
			& Package
			& Subclass
			& World \\
		& & & & \\
		private
			& \ding{51}
			& \ding{56} 
			& \ding{56}
			& \ding{56} \\
		& & & &  \\
		default
			& \ding{51}
			& \ding{51}
			& \ding{56}
			& \ding{56} \\
		& & & & \\
		protected
			& \ding{51}
			& \ding{51}
			& \ding{51}
			& \ding{56} \\
		& & & & \\
		public
			& \ding{51}
			& \ding{51}
			& \ding{51}
			& \ding{51}
	\end{tabular}
\end{table}

\begin{itemize}
	\item \lstinline{public}
		\begin{itemize}
			\item sichtbar innerhalb der eigenen Klasse
			\item sichtbar in Methoden abgeleiteter Klassen
			\item sichtbar für Benutzer der Instanzen
			\item bilden das Interface einer Klasse
		\end{itemize}
	\item \lstinline{private}
		\begin{itemize}
			\item sichtbat innerhalb der eigenen Klasse
			\item nicht sichtbar für abgeleitete Klassen
			\item nicht sichtbar für Benutzer der Instanzen
			\item Grundlage für \textit{Information Hiding}
		\end{itemize}
	\item \lstinline{protected}
		\begin{itemize}
			\item sichtbar innerhalb der eigenen Klasse
			\item sichtbar in Methoden abgeleiteter Klassen
			\item sichtbar in Klassen des selben 
				\lstinline{package}
			\item vor Zugriffen von aussen geschützt ???
		\end{itemize}
	\item \lstinline{defualt}
		\begin{itemize}
			\item sichtbar innerhalb der eigenen Klasse
			\item sichtbar innerhalb des selben 
				\lstinline{package}
		\end{itemize}
\end{itemize}

\noindent



