% coding:utf-8

%kugprg, a LaTeX-Code for a summary of programming modules
%Copyright (C) 2013, Daniel Winz, Ervin Mazlagic

%This program is free software; you can redistribute it and/or
%modify it under the terms of the GNU General Public License
%as published by the Free Software Foundation; either version 2
%of the License, or (at your option) any later version.

%This program is distributed in the hope that it will be useful,
%but WITHOUT ANY WARRANTY; without even the implied warranty of
%MERCHANTABILITY or FITNESS FOR A PARTICULAR PURPOSE.  See the
%GNU General Public License for more details.
%----------------------------------------

\newpage
\section{Container}
Container können nur Objekte enthalten und niemals primitive
Datentypen (siehe Boxing und Autoboxing).

\subsection{List<E>}

\begin{itemize}
	\item geordnet???
	\item indexiert
	\item zero-based (Index beginnt mit 0)
	\item kann mehrfach das gleiche Objekt enthalten
	\item Mehtoden
		\begin{itemize}
			\item \lstinline{add(E element)}
			\item \lstinline{set(int index, E element)}
			\item \lstinline{E get(int index)}
		\end{itemize}
	\item Iteratorenklassen
		\begin{itemize}
			\item \lstinline{Iterator<E>}
			\item \lstinline{ListIterator<E>}
		\end{itemize}
\end{itemize}

\subsection{Set<E>}

\begin{itemize}
	\item nicht indexiert
	\item nicht geordnet
	\item ein Set (zu Deutsch \textit{Menge}) kann keine zwei gleiche 
		Objekte enthalten (wie eine Menge in der Mathematik)
	\item Mehtoden
		\begin{itemize}
			\item \lstinline{add(E element)}
			\item \lstinline{contains(Object o)}
			\item \lstinline{boolean isEmpty()}
			\item \lstinline{int size()}
		\end{itemize}
	\item Iteratoren
		\begin{itemize}
			\item \lstinline{Iterator<E>}
		\end{itemize}
\end{itemize}

\subsection{Map<K, V>}

\begin{itemize}
	\item Verbindet einen Key mit einem Value (bildet also Paare)
	\item jeder Key kommt nur einmal vor
	\item ein Value kann mehreren Keys zugeordnet sein
	\item drei Schichten:
		\begin{itemize}
			\item Set von Keys
			\item Collection von Values
			\item Set von Paaren (Key, Value)
		\end{itemize}
	\item Methoden
		\begin{itemize}
			\item \lstinline{put(K key, V vlaue)}
			\item \lstinline{V get(Object key)}
			\item \lstinline{int size()}
			\item \lstinline{boolean containsKey(Object Key)}
			\item \lstinline{boolean containsValue(Object Value)}
			\item \lstinline{boolean isEmpty()}
			\item \lstinline{V remove(Object Key)}
			\item \lstinline{Set<K> keySet()}
			\item \lstinline{Collection<V> values()}
		\end{itemize}
	\item Iteratoren
		\begin{itemize}
			\item \lstinline{Set<Map.Entry<K,V>>} \\
				d.h. man bildet aus jedem Key-Value
				Paar ein Objekt und daraus dann ein Set
				durch welches iteriert wird
			\item alternativ Iteration über Collection der
				Values möglich
		\end{itemize}
\end{itemize}

\subsection{ArrayList<E>}
Die \lstinline{ArrayLsit<E>} ist eine Implementierung von 
\lstinline{List<E>} mit flexibler bzw. dynamischer Grösse.

\begin{itemize}
	\item \lstinline{size()}
	\item \lstinline{isEmpty()}
	\item \lstinline{get()}
	\item \lstinline{set()}
	\item \lstinline{iterator()}
	\item \lstinline{listIterator()}
\end{itemize}

\lstinputlisting[title=ArrayList Beispiel]{arrayList.java}
